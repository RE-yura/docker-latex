\documentclass[a4j,11pt]{jarticle}
\usepackage {amsmath,amssymb} % さまざまな数式を用いるための設定
\usepackage [dvipdfmx]{graphicx,color} % 図を取り込んだり、色を利用するための設定
\usepackage{cases}
\usepackage{caption}
\usepackage{comment}
\usepackage{url}
\usepackage{multirow}
\usepackage{bm}
\renewcommand{\baselinestretch}{0.9}
\usepackage[top=20truemm,bottom=20truemm,left=25truemm,right=25truemm]{geometry}
%\usepackage{tikz}
%\usetikzlibrary{calc}
%\usetikzlibrary{positioning}
%\usetikzlibrary{arrows.meta}
% \pagestyle{empty}

\begin{document}

\begin{center}
	\LARGE レポート課題 \\
\end{center}
\vspace{-6mm}
\begin{flushright}
	\Large 所属: \\
	\Large 学籍番号: \\
	\Large 氏名:
\end{flushright}

%\begin {abstract}
%\end {abstract}

%==表紙=============================================================================



%==================================================================================
\vspace{-6mm}
\section{圧電素子への入力電圧から変位までの伝達関数}
\vspace{2mm}

%----------------------------------------------------------------------------------
% \vspace{-3mm}
% \subsection{コPip制御}
% \vspace{-1mm}

ピエゾのモデルより以下の3式が立式できる.
%
\begin{equation}
	\label{vin}
	v_\mathrm{in} = R i + \frac{1}{C}\int{i \mathrm{d}t}
\end{equation}
%
\begin{equation}
	\label{xp}
	x_p = K_p \frac{1}{C}\int{i \mathrm{d}t}
\end{equation}
%
\begin{equation}
	\label{motion}
	M \ddot{x} + C_f \dot{x} + K_f x + \frac{K_s K_c}{K_s + K_c} (x - x_p) = 0
\end{equation}
%
(\ref{vin})~(\ref{motion})式をそれぞれラプラス変換し,(\ref{la_vin})~(\ref{la_motion})式を求める.
%
\begin{equation}
	\label{la_vin}
	V_\mathrm{in} = R I + \frac{I}{Cs}
\end{equation}
%
\begin{equation}
	\label{la_xp}
	X_p = K_p \frac{I}{Cs}
\end{equation}
%
\begin{align}
	(K_s & +K_c)(Ms^2 + C_f s + K_f)X + K_s K_c(X - X_p) = 0 \nonumber \vspace{5mm}                              \\
	     & \therefore \frac{X}{X_p} = \frac{K_s K_c}{(K_s+K_c)(Ms^2 + C_f s + K_f) + K_s K_c}  \label{la_motion} 
\end{align}
%
(\ref{la_vin}),(\ref{la_xp})式より,
\begin{equation}
	\label{la_xp2}
	\frac{X_p}{V_\mathrm{in}} = \frac{K_p}{RCs+1}
\end{equation}
と計算され,(\ref{la_motion}),(\ref{la_xp2})式より伝達関数が以下で求まる.

\begin{equation}
	\label{la_x}
	\frac{X}{V_\mathrm{in}} = \frac{K_p K_s K_c}{(RCs+1)\{(K_s+K_c)(Ms^2 + C_f s + K_f) + K_s K_c\}}
\end{equation}

% \begin{align}
%     \label{}
%     a & = 1 \rm{mm} \\
%       & = 2 \\
%       & = 3
% \end{align}

%==================================================================================
\vspace{+10mm}
\section{VCMへの入力電圧から変位までの伝達関数}
\vspace{2mm}

%----------------------------------------------------------------------------------
% \vspace{-3mm}
% \subsection{コPip制御}
% \vspace{-1mm}

VCMのモデルより以下の3式が立式できる.\cite{intuitive2019}
%
\begin{equation}
	\label{q2_vin}
	v_\mathrm{in} = L \frac{\mathrm{d}i}{\mathrm{d}t} + Ri + K_v \frac{\mathrm{d}x}{\mathrm{d}t}
\end{equation}
%
\begin{equation}
	\label{q2_f}
	f = K_i i
\end{equation}
%
\begin{equation}
	\label{q2_motion}
	(M \ddot{x} + C_s \dot{x} + K_s) x = f
\end{equation}
\vspace{8mm}
%
(\ref{q2_vin})~(\ref{q2_motion})式をそれぞれラプラス変換し,(\ref{q2_la_vin})~(\ref{q2_la_motion})式を求める.
\newpage
\begin{equation}
	\label{q2_la_vin}
	V_\mathrm{in} = L s I + R I + K_vsX
\end{equation}
%
\begin{equation}
	\label{q2_la_f}
	F = K_i I
\end{equation}
%
\begin{equation}
	\label{q2_la_motion}
	(M \ddot{x} + C_s \dot{x} + K_s)X = F
\end{equation}
%
(\ref{q2_la_vin}),(\ref{q2_la_f})式より$I$を消去し,
\begin{equation}
	\label{q2_la_f2}
	F = K_i \frac{V_\mathrm{in} - K_v s X}{Ls + R}
\end{equation}
となる.(\ref{q2_la_motion}),(\ref{q2_la_f2})式より伝達関数が以下で求まる.

\begin{align}
	\label{la_x}
	(M \ddot{x}                        & + C_s \dot{x} + K_s) X = K_i \frac{V_\mathrm{in} - K_v s X}{Ls + R} \\
	\therefore \frac{X}{V_\mathrm{in}} & = \frac{K_i}{(M \ddot{x} + C_s \dot{x} + K_s)(Ls+R) + K_i K_v s}    
\end{align}














% %==================================================================================
% \newpage
% \vspace{-6mm}
% \section*{大問3:サーボ式加速度計}
% \vspace{2mm}


% %------------------------------------------------------------------------------------

% %\begin{thebibliography}{9}
% %
% %\bibitem{1}
% %{マーク・ホジキンソン(1990)『ロジャー・フェデラー』(鈴木佑依子訳)東洋館出版社.}
% %
% %\bibitem{2}
% %{がん治療ブログ「がん診断に使用する画像診断機器・CTとMRIの違いとは?」,2017年10月26日(最終閲覧日:2018年5月13日) \\
% %\url{http://ucc-radiotherapy.com/ct_mri/}}
% %
% %\end{thebibliography}


% %------------------------------------------------------------------------------------

% %\begin{figure}[!h]
% %\begin{center}
% %\includegraphics[width=0.65\linewidth]{result.png}
% %\end{center}
% %\vspace{-5mm}
% %\caption{実験結果}
% %\label{result}
% %\end{figure}

% %\begin{figure}[!h]
% %\begin{minipage}{0.5\hsize}
% %\begin{center}
% %\includegraphics[width=0.5\linewidth]{robo1.png}
% %\end{center}
% %\vspace{-4mm}
% %\caption{歩行ロボット}
% %\label{robo1}
% %\end{minipage}
% %\begin{minipage}{0.5\hsize}
% %\begin{center}
% %\includegraphics[width=0.5\linewidth]{robo2.png}
% %\end{center}
% %\vspace{-4mm}
% %\caption{ロボット}
% %\label{robo2}
% %\end{minipage}
% %\end{figure}

% \begin{table}[!h]
%     \renewcommand{\arraystretch}{1.2}
%     \begin{center}
%         \caption{上の振子と下の振子の一次モードと二次モードの振幅と位相差}
%         \begin{tabular}{|c|c|c|c|c|}\hline
%                        & \multicolumn{2}{|c|}{1次モード} & \multicolumn{2}{|c|}{2次モード}                         \\ \cline{2-5}
%                        & 振幅(deg)                       & 位相(rad)                       & 振幅(deg) & 位相(rad) \\ \hline \hline
%             $\theta_1$ & 9.148                           & 1.600                           & 3.060     & -0.2023   \\ \hline
%             $\theta_2$ & 6.610                           & 1.777                           & 8.122     & 2.817     \\ \hline
%         \end{tabular}
%         \label{振幅と位相差}
%     \end{center}
% \end{table}

% %\begin{table}[!h]
% %\renewcommand{\arraystretch}{1.2}
% %\begin{center}
% %\caption{対策1による技術面およびユーザの心理面における長所と短所}
% %\begin{tabular}{|c|c|l|}\hline
% % \multirow{2}{*}{技術面} & 長所 & 変更が容易 \\ \cline{2-3}
% %										& 短所 & カメラの位置の変更も必要 \\ \hline
% %\multirow{2}{*}{心理面} & 長所 & 口から声が聞こえてこない不快感がなくなる \\ \cline{2-3}
% %										& 短所 & なし \\ \hline
% %\end{tabular}
% %\label{taisaku1}
% %\end{center}
% %\end{table}

% \begin{itemize}
%     \setlength{\itemsep}{0.5mm} % 項目の隙間
%           \setlength{\parskip}{0mm} % 段落の隙間
%     \item a
%     \item b
% \end{itemize}

% \begin{numcases}
%     {}
%     脚の慣性モーメントI = \int_0^L L'^2(\rho 2 \pi L' dL')=\dfrac{\pi}{2}\rho L^4=\dfrac{1}{2}ML^2 \label{mom} \\
%     角速度\omega=\dfrac{V}{L} \label{7}
% \end{numcases}

% \begin{eqnarray}
%     \left\{
%     \begin{array}{ll}
%         T_H=150^\circ\mathrm{C} \\
%         T_L=69.9^\circ\mathrm{C}
%     \end{array} \right.
%     \label{1}
% \end{eqnarray}

% \begin{align}
%     \label{4}
%     \begin{pmatrix}
%         X_p \\
%         Y_p
%     \end{pmatrix} & =
%     \begin{pmatrix}
%         X_1 \\
%         Y_1
%     \end{pmatrix}+
%     \begin{pmatrix}
%         \cos\phi & -\sin\phi \\
%         \sin\phi & \cos\phi
%     \end{pmatrix}・
%     \overrightarrow{J_1P} \nonumber \\
%                               & =
%     \begin{pmatrix}
%         X_1 \\
%         Y_1
%     \end{pmatrix}+
%     \begin{bmatrix}
%  	    \cos\phi & -\sin\phi \\
%         \sin\phi & \cos\phi
%     \end{bmatrix}
% %    \left[
% %        \begin{array}{rr}
% %            \cos\phi & -\sin\phi \\
% %            \sin\phi & \cos\phi
% %        \end{array}
% %    \right]・
%     \begin{pmatrix}
%         \xi \\
%         \eta
%     \end{pmatrix}^{\mathrm{T}}
% \end{align}

% \begin{equation}
%     \label{5}
%     「P_c>P_h」\iff n>\dfrac{Q\times P}{25\times\dfrac{92}{100}\div24\times 1.163}\approx 0.99 \ \mathrm{mm}
% \end{equation}

% \fancyhead[CO]{参考文献}
\bibliographystyle{junsrt}
\bibliography{./refs}

\end{document}
